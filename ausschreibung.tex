% TODO: Ausschreibung als HTML auf Homepage

\documentclass[paper=a4, fontsize=10pt]{scrartcl}


\usepackage{blitzmarathon}
\usepackage{parameters}


\begin{document}


\tournamentHeader{img/logo_sparkasse}
                 {img/logo_sk_ettlingen}
                 {\tournamentName}
                 {tournamentDate}


\begin{basedescript}{\desclabelstyle{\multilinelabel}\desclabelwidth{10em}}


\item[Veranstalter:]
  \organizerName{} mit freundlicher Unterstützung der \sponsorName{}

\item[Spielplattform:]

  Online auf \url{\lichessMainPageURL}. \\
  Account erstellbar unter: \url{\lichessSignupURL}

  Live-Ergebnisse auf \url{\tournamentLiveURL}.

\item[Spielberechtigt:]

  Europäische Spieler mit FIDE-ID sowie Mitglieder in einem dem DSB
  über ihre Landesverbände angeschlossenen Verein.

\item[Modus:]

  Voraussichtlich ca. \expectedTotalGames{} Partien. Zunächst wird
  eine gemeinsame Vorrunde, anschließend die Finalrunde, getrennt in
  A- und B-Gruppe, gespielt. Punkte aus der Vorrunde werden zu 50\% in
  die Finalrunde mitgenommen. Der Modus wird ggf.\ entsprechend der
  Teilnehmerzahl angepasst.

\item[Bedenkzeit:]

  \timePerGame{} pro Partie (\pauseBetweenGames{} Pause zwischen den
  Partien)

\item[Anmeldung:]

  Auf \url{\tournamentURL} per Webformular oder per Email an
  \email{\tournamentEmail}.

\item[Preise:]

  \hspace{2em}\begin{tabular}[t]{ll}
    Hauptpreise:       &  \is{/}{\prizesTournamentA} Euro \\
    B-Gruppe:          &  \is{/}{\prizesTournamentB} Euro \\
    Sonderpreise:      &  \is{/}{\specialPrizesCateries} je \specialPrizes{} Euro \\
    Jugendpreis (U18)  &  \is{/}{\youthPrizes} Euro \\
    Ratingpreise:      &  DWZ \is{/}{\ratingPriceCategories} je \ratingPrices{} Euro
  \end{tabular}

  Die jeweils ersten Preise im A- und B-Open sind garantiert, die
  Gesamtausschüttung aller Preise ab
  \prizesGuaranteedMinParaticipants{} zahlenden Teilnehmern.

  Vollständige Ausschüttung der Sonder-, Rating- und Jugendpreise erst
  ab \specialPrizesGuaranteedMinParaticipants{} zahlenden Teilnehmern
  der jeweiligen Kategorie. Doppelpreise sind ausgeschlossen.

\item[Startgeld:]

  Das Startgeld beträgt (maßgeblich ist der Zahlungseingang):

  \hspace{2em}\begin{tabular}[t]{ll}
  für die ersten 20 Meldungen:                             &  6 Euro \\
  bis \dateDiff{tournamentDate}{-7}:  &  7 Euro \\
  bis 10:00 Uhr am Spieltag:   &  8 Euro \\
  danach:                                                  &  10 Euro \\
  Jugendliche U18 erhalten 2 Euro Rabatt.
  \end{tabular}

\item[Zeitplan am \DTMUsedate{tournamentDate}:]
  \hspace{2em}\begin{tabular}[t]{ll}
    bis 13:45 Uhr   &  Eintreffen im Turnierraum \\
    14:00 Uhr       &  Start Vorrunde (ca. 21 Runden) \\
    ca. 18:00 Uhr   &  Pause \\
    bis 18:45 Uhr   &  Eintreffen im Turnierraum der Finalrunden \\
    19:00 Uhr       &  Start Finalrunde (ca. 15 Runden) \\
    ca. 22:00 Uhr   &  Ende Turnier
  \end{tabular}

  Alle Spieler müssen sich am Turniertag pünktlich bis 13:45 Uhr in
  der Lichess Turniergruppe
  \href{\lichessTournamentTeamURL}{``\tournamentName``} (Freischaltung
  notwendig) einfinden.

  Nach der Vorrunde erfolgt die Gruppeneinteilung in A- und B-Finale
  und wird auf der Homepage bekannt gegeben. Für das A-Finale wird an
  die qualifizierten Teilnehmer ein Kennwort zur Teilnahme im
  Lichess-Turnier des A-Finale verschickt. Alle Teilnehmer müssen sich
  15 min vor Start in den jeweiligen Endrundengruppen einfinden.

\item[Bezahlung:]
  Zahlung entweder per Banküberweisung an

  \hspace{2em}\begin{tabular}[t]{ll}
    IBAN: & \bankFormat{\IBAN} \\
    BIC:  & \bankFormat{\BIC} \\
          & \bankFormat{\bank}
  \end{tabular}

  oder per PayPal an \texttt{\paypalEmail}.

  Bei Anmeldung nach dem \dateDiff{tournamentDate}{-3}ist das
  Startgeld ausschließlich per PayPal zu zahlen.

\item[Fair Play:]

  Es gelten die \href{\lichessTermsOfServiceURL}{Lichess Fair Play and
    Community Guidelines}. Insbesondere ist es auch nicht erlaubt,
  sich beim Spielen abzuwechseln oder im Team zu spielen. Bei
  Verstößen liegt die endgültige Entscheidung (über
  Disqualifizierungen) beim Veranstalter.

\item[Schiedsrichter:]

  \arbiter{}

\item[Streaming:]

  Links auf Turnierberichte oder Livestreams werden von uns gerne auf
  der Homepage veröffentlicht bzw. eingebunden. Dazu bitten wir vorab
  um eine kurze Mitteilung.

\item[weitere Hinweise:]

  Bei Punktgleichheit in der Gesamtwertung (bzw. in der Vorrunde)
  entscheidet:

  \enum{\tiebreak}

  Die Auszahlung der Preise erfolgt ausschließlich per
  SEPA-Überweisung innerhalb von 2 Wochen nach
  Turnierende. Preisträger müssen nach dem Turnier den Organisatoren
  ihre IBAN mitteilen, um die Auszahlung zu ermöglichen.

  Wichtige Hinweise der Turnierleitung sind auf der
  \href{\tournamentURL}{\texttt{Homepage}} einzusehen
  (Gruppeneinteilung). Die Ergebnisse werden mit Klarnamen auf der
  \href{\tournamentURL}{\texttt{Homepage}} live veröffentlicht.

  Die Turnierleitung behält sich Änderungen in Turnierangelegenheiten
  vor.

\item[Ansprechpartner, Turnierleitung und Infos:]

  Simon Fromme und Lennard Löwe, erreichbar per Email unter
  \\ \email{\tournamentEmail} oder über den Lichess-Chat an
  \lichessNameLink{sifro}, \lichessNameLink{Ralinga} und
  \lichessNameLink{Thomas\_Batton}

  Weitere Informationen und Anmeldung auf der
  \href{\tournamentURL}{Turnier-Homepage}
\end{basedescript}

\end{document}
