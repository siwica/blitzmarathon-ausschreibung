\documentclass[paper=a4, fontsize=10pt]{scrartcl}

\usepackage{geometry} \geometry{a4paper, top=20mm, left=35mm,
  right=25mm, bottom=15mm, headsep=20mm, footskip=0mm}

\usepackage{blitzmarathon}
\usepackage{parameters}


\begin{document}


\tournamentHeader{img/logo_sparkasse}
                 {img/logo_sk_ettlingen}
                 {\tournamentName}
                 {tournamentDate}


\begin{description}[leftmargin=\dimexpr\wd\descbox+\labelsep,align=Left]

\item[Veranstalter:]
  \organizerName{} mit freundlicher Unterstützung der \sponsorName{}

\item[Spielort:]

  Online auf \url{\lichessMainPageURL} (Account erstellbar unter
  \url{\lichessSignupURL})

\item[Spielberechtigt:]

  Mitglieder (aktiv/passiv) in einem dem DSB über ihre Landesverbände
  angeschlossenen Verein sowie europäische Spieler mit FIDE-ID.

\item[Zeitplan am \DTMUsedate{tournamentDate}:]

  Alle Spieler müssen sich voranmeldet am Turniertag pünktlich bis
  13:45 Uhr in der Lichess Turniergruppe
  ``\tournamentName'' (\url{\lichessTournamentTeamURL},
  Freischaltung notwendig) einfinden.

\item[Modus:]

  Voraussichtlich ca. \expectedTotalGames{} Partien. Zunächst wird
  eine gemeinsame Vorrunde, anschließend die Finalrunde, getrennt in
  A- und B-Gruppe, gespielt. Der Modus wird ggf.\ entsprechend der
  Teilnehmerzahl angepasst.

\item[Bedenkzeit:]

  \timePerGame{} pro Partie (\pauseBetweenGames{} Pause zwischen den
  Partien)

\item[Fair Play:]

  Es gelten die \href{\lichessTermsOfServiceURL}{Lichess Fair Play and
    Community Guidelines}. Insbesondere ist es auch nicht erlaubt,
  sich beim Spielen abzuwechseln oder im Team zu spielen. Bei
  Verstößen liegt die endgültige Entscheidung (über
  Disqualifizierungen) beim Veranstalter.

\item[Schiedsrichter:]

  \arbiter{}

\item[Preise:]

  \begin{tabular}{ll}
    Hauptpreise:       &  \is{/}{\prizesTournamentA} Euro \\
    B-Gruppe:          &  \is{/}{\prizesTournamentB} Euro \\
    Sonderpreise:      &  \is{/}{\specialPrizesCateries} je \specialPrizes{} Euro \\
    Ratingpreise:      &  DWZ \is{/}{\ratingPriceCategories} je \ratingPrices{} Euro \\
    bester Streamer:   &  \specialPrizes{} Euro
  \end{tabular}

  Die ersten beiden Preise sind garantiert, die Gesamtausschüttung
  aller Preise ab \prizesGuaranteedMinParaticipants{} zahlenden
  Teilnehmern.

  Sonder-, Rating- und Jugendpreise erst ab
  \specialPrizesGuaranteedMinParaticipants{} zahlenden Teilnehmern der
  jeweiligen Kategorie. Doppelpreise sind ausgeschlossen.

  Die Auszahlung der Preise erfolgt ausschließlich per
  SEPA-Überweisung innerhalb von 2 Wochen nach Turnierende.

\item[Anmeldung Startgeld:]

  Anmeldung auf \url{\tournamentURL} per Webformular oder per Email an
  \\ \email{\tournamentEmail}.

  Das Startgeld beträgt:

  \begin{tabular}{lr}
  für die ersten 20 Meldungen: 6 Euro \\
  bei Startgeldeingang bis \dateDiff{tournamentDate}{-14}: 8 Euro \\
  bei Anmeldung und Bezahlung bis 10:00 Uhr am Spieltag (\dateDiff{tournamentDate}{0}): 10 Euro \\
  danach: 12 Euro \\
  Jugendliche U18 erhalten 2 Euro Rabatt.
  \end{tabular}

  Zahlung entweder per Banküberweisung an:

  \begin{tabular}{ll}
    IBAN: & \bankFormat{\IBAN} \\
    BIC:  & \bankFormat{\BIC} \\
          & \bankFormat{\bank}
  \end{tabular}

  oder per PayPal an \texttt{\paypalEmail}. Bei Anmeldung nach dem
  \dateDiff{tournamentDate}{-3} ist das Startgeld ausschließlich per
  PayPal zu zahlen.

\item[Streaming:]

  Links auf Turnierberichte oder Livestreams werden von uns gerne auf
  der Homepage veröffentlicht bzw. eingebunden.

\item[weitere Hinweise:]

  Für die Gesamtwertung zählen die erzielten Punkte aus der Vorrunde
  einfach und aus der Endrunde doppelt.  Bei Punktgleichheit in der
  Gesamtwertung entscheidet:

  \enum{\tiebreak}

  Wichtige Hinweise der Turnierleitung sind auf der
  \href{\tournamentURL}{\texttt{Homepage}} einzusehen
  (Gruppeneinteilung). Die Ergebnisse werden mit Klarnamen auf der
  \href{\tournamentURL}{\texttt{Homepage}} live veröffentlicht.

  Die Turnierleitung behält sich Änderungen in Turnierangelegenheiten
  vor.

\item[Ansprechpartner, Turnierleitung und Infos:]

  Simon Fromme und Lennard Löwe erreichbar per Email unter
  \\ \email{\tournamentEmail} oder über den Lichess-Chat an
  \lichessNameLink{sifro}, \lichessNameLink{Ralinga} und
  \lichessNameLink{Thomas\_Batton}

  Weitere Informationen und Anmeldung auf der Turnier-Homepage.

  Nach der Vorrunde erfolgt die Gruppeneinteilung in A- und B-Finale
  und wird auf der Homepage bekannt gegeben. Für das A-Finale wird an
  die qualifizierten Teilnehmer ein Kennwort zur Teilnahme im
  Lichess-Turnier des A-Finale verschickt. Eine Weitergabe des
  Passworts an andere Spieler ist nicht gestattet. Alle Teilnehmer
  müssen sich 15 min vor Start in den jeweiligen Endrundengruppen
  einfinden.
\end{description}

\end{document}
